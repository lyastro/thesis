\begin{resume}

  \resumeitem{个人简历}

  1990 年 06 月 17 日出生于 青海 省 格尔木 市。

  2009 年 08 月考入 南京 大学 天文 系 天文学 专业,2013 年 7 月本科毕业并获得 理学 学士学位。

  2013 年 08 月免试进入 清华 大学 物理系 系攻读 博士 学位至今。

  \researchitem{发表的学术论文} % 发表的和录用的合在一起

  % 1. 已经刊载的学术论文(本人是第一作者,或者导师为第一作者本人是第二作者)
  \begin{publications}
    \item Liang Y, Zhao C, Chuang C H, et al. Measuring baryon acoustic oscillations from the clustering of voids. MNRAS, 2016, 459:4020–4028. (SCI 收录, ISD:DR3VM, WOS:000379830700046)
  \end{publications}

  % 2. 尚未刊载,但已经接到正式录用函的学术论文(本人为第一作者,或者
  %    导师为第一作者本人是第二作者)。
%  \begin{publications}[before=\publicationskip,after=\publicationskip]
%    \item Yang Y, Ren T L, Zhu Y P, et al. PMUTs for handwriting recognition. In
%      press. (已被 Integrated Ferroelectrics 录用. SCI 源刊.)
%  \end{publications}

  % 3. 其他学术论文。可列出除上述两种情况以外的其他学术论文,但必须是
  %    已经刊载或者收到正式录用函的论文。
  \begin{publications}
    \item Kitaura F S, Chuang C H, Liang Y, et al. Signatures of the Primordial Universe from Its Emptiness: Measurement of Baryon Acoustic Oscillations from Minima of the Density Field. Physical Review Letters, 2016, 116(17):171301. (SCI 收录, ISD:DK5MI, WOS:000374963600003)
    \item Zhao C, Tao C, Liang Y, et al. DIVE in the cosmic web: voids with Delaunay triangulation from discrete matter tracer distributions. MNRAS, 2016, 459:2670–2680. (SCI 收录, ISD:DR3ZJ, WOS:000379840900031)
    \item Dawson K S, Kneib J P, Percival W J, et al. The SDSS-IV Extended Baryon Oscillation Spectroscopic Survey: Overview and Early Data. AJ, 2016, 151:44. (SCI 收录, ISD:DF3MO, WOS:000371248600024)
  \end{publications}

\end{resume}
