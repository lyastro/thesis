% 如果使用声明扫描页,将可选参数指定为扫描后的 PDF 文件名,例如:
% \begin{acknowledgement}[scan-statement.pdf]
\begin{acknowledgement}
  衷心感谢 陶嘉琳 教授从2011年开始对本人的精心指导与帮助,并接收我保送到清华直接攻读博士学位。在学术上 陶嘉琳 教授严谨求实的态度深深的影响了我,使我时刻思考着自己科研工作背后的科学意义,并慎重的关注着每个结果的误差与可靠性。在生活中, 陶嘉琳 教授是位慈祥的长者,用心呵护着身边每一个人,使周围所有的人都从她身上感受到阳光般的温暖,沐浴着人性的光辉,而我这些年非常荣幸在她的身边蒙受恩泽。

  感谢导师 莫厚俊 教授对本人的精心指导,每次与 莫厚俊 教授的交流都使我颇受启发茅塞顿开,读arxiv每日最新论文时 莫厚俊 教授对每篇工作的点评、对作者趣事的娓娓道来都使我领略天文学领域一代宗师的气派。

  感谢Leibniz-Institut f\"ur Astrophysik Potsdam (AIP)的博士后 Chia-Hsun Chuang 博士从2013年暑期开始一直以来的合作与指导。 Chia-Hsun Chuang 博士无私的分享了他在BOSS/SDSS-III项目中的知识与经验,在工作中制定详细周全的计划,在困难中给予鼓励并与一起并肩作战。从他身上我学到作为一个合作者的典范。

  感谢Universidad de La Laguna(ULL)和Instituto de Astrofisica de Canarias(IAC)的 Francisco-Shu Kitaura 教授,感谢他在2015年马德里的SDSS合作者会议与我的讨论中鼓励我研究巨洞的重子声波振荡,并在后续的研究中不断给予非常重要的支持和帮助。他的聪明才智与对物理深厚的理解使我受益匪浅。

  感谢 赵成 同学在合作中的鼎力支持以及几年来同窗之谊。

  感谢 清华大学物理系天体物理中心 主任 毛淑德 教授,以及天体物理中心全体老师和同学们、清华大学物理系和清华大学的帮助和支持。

  感谢爱人 李玉静 和家人对我的关怀与支持,你们是我最坚强的后盾。

  在Instituto de Astrofisica de Canarias(IAC)进行一个月的合作研究期间,承蒙 Juan E. Betancort-Rijo 教授热心指导与帮助,使我对巨洞在宇宙中的演化以及研究巨洞的方式有了更深厚的了解。期间IAC的博士生Marcos Pellejero Ibanez带我领略了加那利群岛的美妙风光,不胜感激。

  感谢“天文学名词网站”\footnote{\url{http://www.lamost.org/astrodict/index.php}}提供的学术名词翻译。

  本工作承蒙清华大学985基金、国家科技部973项目(2013CB834906)、国家自然科学基金(11033003和11173017)的资助,特此致谢。同时我们感谢 Sino French CNRS-CAS international laboratories LIA Origins 、 FCPPL 、 Barcelona (MareNostrum) 、 LRZ (Supermuc) 、AIP (erebos) 、 CCIN2P3 (Quentin Le Boulc’h) 和清华大学提供的计算资源。

  SDSS-III项目由Alfred P. Sloan Foundation 、the Participating Institutions 、 the National Science Foundation 和 the US Department of Energy Office of Science 资助。SDSS-III 的网站为 http://www.sdss3.org/。SDSS-III项目由the Astrophysical Research Consortium for the Participating Institutions of the SDSS-III Collaboration(包括 University of Arizona 、 the Brazilian Participation Group 、 Brookhaven National Laboratory 、 Carnegie Mellon University 、 University of Florida 、 the French Participation Group 、 the German Participation Group 、 Harvard University 、 Instituto de Astrofisica de Canarias 、 the Michigan State/Notre Dame/JINA Participation Group 、 Johns Hopkins University 、 Lawrence Berkeley National Laboratory 、 Max Planck Institute for Astrophysics 、 Max Planck Institute for Extraterrestrial Physics 、 NewMexico State University 、 New York University 、 Ohio State University 、 Pennsylvania State University 、 University of Portsmouth 、 Princeton University 、 the Spanish Participation Group 、 University of Tokyo 、 University of Utah 、 Vanderbilt University 、 University of Virginia 、 University of Washington 和 Yale University)管理运作。

  感谢 \thuthesis,它的存在让我的论文写作轻松自在了许多,让我的论文格式规整漂亮了许多。

\end{acknowledgement}
