\chapter{引言}
\label{cha:intro}

在十年前,通过计算斯隆数字化巡天~\cite{York2000} (Sloan Digital Sky Survey,SDSS) DR3 亮红星系(Luminous Red Galaxy,LRG)~\cite{Eisenstein2001}样本的两点相关函数(two-point correlation function, 2PCF)和 2度视场星系红移巡天~\cite{Colless2001}(Two Degree Filed Galaxy Redshift Survey,2dFGRS)星系样本的功率谱(power spectrum),重子声波振荡(baryon acoustic oscillation,BAO)首次被直接探测到~\cite{Eisenstein2005,Cole2005}。
此后SDSS的巡天观测以及其他星系红移巡天(galaxy redshift surveys)项目的成果大大提高了对BAO特征尺度测量的精确性~\cite{Percival2010,Blake2011,Beutler2011,Drinkwater2010,White2011,Eisenstein2011,Dawson2013,Anderson2014441,Ross2015}。
重子振荡光谱巡天~\cite{Dawson2013}(Baryon Oscillation Spectroscopic Survey,BOSS)是斯隆数字化巡天三期~\cite{Eisenstein2011}(SDSS-III)的重要组成部分,该项目首次成功利用莱曼$\alpha$森林(Lyman $\alpha$ forest)的自相关函数(auto-correlation function),和莱曼$\alpha$森林与高红移类星体(Quasar)的互相关函数(cross-correlation function)探测到BAO信号~\cite{Busca2013,Slosar2013,Font-Ribera2014,Delubac2015}。

因为BAO的特征尺度受系统误差影响较小,能在较高红移被精确测量,其大小可以从第一原理(first principle)出发用基本的宇宙学参数计算得出,所以BAO成为精确测量宇宙膨胀历史和研究宇宙大尺度结构形成演化的最好探针之一。目前将BAO和宇宙微波背景辐射(cosmic microwave background,CMB)结合起来对宇宙学参数的限制能力已经超越Ia型超新星(type Ia supernova,SNIa)与CMB的组合~\cite{ABB14}。
未来的几个大型巡天项目和望远镜,包括The Dark Energy Spectroscopic Instrument~\cite{bigboss2011}(DESI)、The Dark Energy Survey~\cite{des2013}(DES)、大口径全天巡视望远镜~\cite{lsst2012}(The Large Synoptic Survey Telescope,LSST)、The Javalambre Physics of The Accelerating Universe Astrophysical Survey~\cite{jpas2014}(J-PAS)、The 4-metre Multi-Object Spectroscopic Telescope~\cite{4most}(4MOST)、欧几里得空间望远镜~\cite{euclid2009}(EUCLID)、Subaru Prime Focus Spectroscopy~\cite{Tamura2016}(PFS)和大视场红外巡天望远镜\footnote{\url{http://wfirst.gsfc.nasa.gov}}(Wide-Field Infrared Survey Telescope,WFIRST),都把精确测量BAO特征尺度作为其重要的科学目标之一。

巨洞是宇宙大尺度结构中占据大部分空间的低密度区域,里面一般不存在大质量星系。The Giant Bo\"otes voids~\cite{KOS81}是首次被观测到的巨洞。随后的星系巡天观测中,如CfA红移巡天~\cite{LGH86,VGP94}(The Center of Astrophysics redshift survey)、红外天文卫星~\cite{EP97}(The Infrared Astronomical Satellite,IRAS)、拉斯坎帕纳斯红移巡天~\cite{MAE00}(The Las Campanas Redshift Survey,LasCampanas)、点源星表红移巡天~\cite{PB02}(The IRAS Point Source Catalog Redshift Survey,PSCz)、2dFRGS~\cite{CCG04,HV04,Patiri2006HB}、The Deep Extragalactic Evolutionary Probe 2~\cite{CCW05}(DEEP2)、The 2MASS Redshift Survey~\cite{NKH14}(2MRS)、SDSS~\cite{Platen2011,VBT12,PVH12,NH14,SLW14,BKH15}和The Visible Multi-Object Spectrograph~\cite{Micheletti02014}(VIMOS),越来越多的巨洞被观测到。巨洞的存在显示了宇宙在大尺度上并不非常均匀,而是以网状的结构(cosmic web structure)存在,巨洞是宇宙网状结构中很明显的一部分。

基于暗物质密度场的宇宙学演化,可以从理论的角度研究巨洞~\cite{Doroshkevich1970,Zeldovich1970,Hahn2007}。而通过宇宙学大尺度结构数值模拟也可以证明巨洞的存在和演化~\cite{SW04,Colberg2005, 2006MNRAS.367.1629S, Platen2007, Neyrinck2008, Forero-Romero2009, 2012MNRAS.425.2049H, 2013MNRAS.429.1286C}。巨洞是限制宇宙学参数或检验不同引力理论的探针。巨洞的数密度(number density)分布和概率分布函数(probability distribution functions,PDF)可以用来限制$\sigma_8$和$\Omega_m h$~\cite{BPP09,2015arXiv151104075S}。巨洞的统计性质还可以用作其他宇宙学研究~\cite{W79,PP86,B90,EEG91,BL02}。巨洞堆叠起来之后的形状可以用阿尔科克-帕金斯基检验(Alcock-Paczy$\acute{\rm n}$ski test,AP test)限制暗能量模型~\cite{PL07,LW10,Pisani2015PRD},巨洞的形状还可以检验动力学暗能量模型~\cite{B12}(dynamical dark energy model)和耦合暗能量模型~\cite{L11}(coupled dark energy model)。巨洞的统计性质还可以用来检验不同的引力理论~\cite{MS09,LZK12,CC13,LCC15,Cai2015}或研究萨克斯-沃尔夫效应(The Sachs-Wolfe effect,SZ 效应)~\cite{GNS08,ILD13,Cai2014,HNG15,PLANCKISW14}。

曾有人研究过巨洞的两点相关函数,但是因为那些研究中巨洞的数密度非常低,所以用那些定义巨洞的方法不可能探测到有效的BAO信号~\cite{Padilla2005,Patiri2006372,VBT12,CC13,CJS15}。在我们的研究中引入了一种全新的定义巨洞的方式:对在三维空间离散分布的星系或暗物质晕(dark matter halo)做德劳内三角剖分(Delaunay Triangulation,DT)而得到的四面体所确定的空心球定义为德劳内三角剖分巨洞(DT巨洞)。同时我们开发了一款能够非常高效的寻找这些DT巨洞的工具DIVE(Delaunay trIangulation Void findEr)。用DIVE所得到的DT巨洞可以让我们有机会获得宇宙低密度区域的位置并研究其成团性,而用这种定义巨洞的方式能否探测到一个高置信度的BAO信号,以及通过DT巨洞探测到的BAO信号相对于直接用星系成团性所测的BAO信号有什么区别将成为本文的主要研究内容。直觉上,相比星系或星系团,巨洞所遭受的引力作用应该比较小,因此巨洞的中心理应受非线性演化影响较小,巨洞的体积在演化中会以球对称的方式变大~\cite{SW04}。所以巨洞应该是一种更加理想的用以研究宇宙最初的密度涨落的探针。

本文的结构如下:
第~\ref{cha:bao} 章综述了用星系测量BAO的研究;
%第~\ref{cha:void} 章总结了不同的寻找巨洞的方式及相应的研究;
第~\ref{cha:data} 章介绍了本文研究中所使用的数据;
第~\ref{cha:DIVE} 章展示了DIVE及DT巨洞的统计性质;
第~\ref{cha:voidbao} 章研究了如何从DT巨洞中探测BAO信号;
%第~\ref{cha:recon} 章讨论了BAO信号重构分别对星系和巨洞BAO信号的影响;
第~\ref{cha:summary} 章总结了全文的主要结论。

